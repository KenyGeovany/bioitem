% BIOGRAFIAS (07.08.2024)

%Necesary commands
\usepackage{hyperref}
\usepackage{datatool}
\usepackage{xcolor}

% Define the data base.
\DTLnewdb{biography}

%DINAMIC BIOGRAFY
% Definir el comando personalizado \mylabel
\newcommand{\biolabel}[1]{%
  \expandafter\gdef\csname bio@label@#1\endcsname{}%
}
% Define a command named \bioitem to input the data in rows with 5 columns.
\newcommand{\bioitem}[5]{%
  \DTLnewrow{biography}% Debe coincidir con alguna base de datos
  \DTLnewdbentry{biography}{label}{#1}%
  \DTLnewdbentry{biography}{name}{#2}%
  \DTLnewdbentry{biography}{birth}{#3}%
  \DTLnewdbentry{biography}{death}{#4}%
  \DTLnewdbentry{biography}{description}{#5}%
  \label{LabelUniqueFromPackageBiographyBioItem#1}%
  % Label para verificar si una referencia apunta a una
  % biografía existente.
}
% Command to make hyperlinks with control of labels
\newcommand{\biohyperlink}[2]{%
% Se verifica primero si el label está definida para algún bioitem
\ifcsname r@LabelUniqueFromPackageBiographyBioItem#1\endcsname
    % Label personalizado para imprimir un label si esta etiqueta existe
    % \biolabel no presenta ningún warning por duplicación
    \biolabel{LabelUniqueFromPackageBiography#1}%
    \hyperlink{#1}{\textcolor{purple}{#2}}%
\else
    \PackageWarning{biography}{Biography Label does not exists}%
\fi
}
\newcommand{\printBiography}[0]{%
  \newcount\countreferenced
  \countreferenced=0
  \DTLforeach{biography}{\label=label}{%
    \ifcsname bio@label@LabelUniqueFromPackageBiography\label\endcsname
      \advance\countreferenced by 1
    \fi
  } % Check first if some reference there exists
  \ifnum\countreferenced>0
    \begin{description}
    \DTLsort{name}{biography}% Sorter the biographies
    \DTLforeach{biography}{\label=label, \name=name, \birth=birth, \death=death, \description=description}{%
      \ifcsname bio@label@LabelUniqueFromPackageBiography\label\endcsname
        \item[\name \hspace{1mm}(\birth \hspace{1mm} - \death)]%
        \hypertarget{\label}{} \description
      \fi
    } % If there is some reference then print the biography
    \end{description}
  \else
    % If not, print a warning
    \PackageWarning{biography}{None biography referenced can be found}
  \fi
}

%COMMENTS
%Make a section for biography: \section{Biographies} (optional)
%Cite with: \biohyperlink{label}{name}
%Define the entries of the data base in the \document with:
%\bioitem{label}{name}{birth}{date}{description}
%Change the sorting in the definition of \printBiography: 
%\DTLsort{type}{biography}, type = name, label, etc.
%Print the biographies: in \document write \printBiography 
%I suggest to split the entries of the data base in a separte file, 
%e.g. biography.tex
%I further suggest define a universal file biography.tex 
%and print the referenced entries only.
