% BIOGRAFIAS

% Required packages
\usepackage{hyperref}
\usepackage{datatool}

% Define the data base.
\DTLnewdb{biography}

% STYLE 1: STATIC BIOGRAFY
% Define a command named \bioitem to input the data in rows with 5 columns.
\newcommand{\bioitem}[5]{%
  \DTLnewrow{biography}% Must coincide with some data base.
  \DTLnewdbentry{biography}{label}{#1}%
  \DTLnewdbentry{biography}{name}{#2}%
  \DTLnewdbentry{biography}{birth}{#3}%
  \DTLnewdbentry{biography}{death}{#4}%
  \DTLnewdbentry{biography}{description}{#5}%
}
% Print the biography sorted by name
\newcommand{\printBiography}{%
\begin{description}
\DTLforeach{biography}{\label=label, \name=name, \birth=birth, \death=death, \description=description}{%
  \item[\name \hspace{1mm}(\birth \hspace{1mm}- \death)]\hypertarget{\label}{} \description
}
\end{description}%
}



%Make a section for biography: \section{Biographies} (optional)
%Cite with: \biohyperlink{nameLabel}{lastName}
%Define the entries of the data base in the \document with:
%\bioitem{nameLabel}{name}{birth}{date}{description}
%Change the type of sort in the definition of \printBiography: 
%\DTLsort{type}{biography}, type = name, label, etc.
%Print the biographies: in \document write \printBiography 
%I suggest to split the entries of the data base in a separte file, 
%e.g. biographies.tex
%I further suggest define a universal file biographies.tex 
%and print the referenced entries only.

