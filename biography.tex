% Aquí van las biografías de Matemáticos.
% Este es un script universal, por lo que debe estar contenido en latexlib.
\bioitem{bioWilson}{John Wilson}{1741}{1796}{%(07.08.2024)
Matemático y Juez británico. Nació en Applethwaite, Cumbría, Inglaterra. Fue alumno de Edward Waring en el colegio universitario de Peterhouse de Cambridge. Fue senior Wrangle en 1761. En 1766 comenzó a ejercer como abogado. Compaginó su carrera como Juez con su actividad en el campo de las matemáticas. Fue nombrado caballero y se convirtió en miembro de la Royal Society en 1782. Obra:
}
\bioitem{bioWaring}{Edward Waring}{1734}{1798}{%(07.08.2024)
Matemático inglés nacido en Old Heath, Inglaterra. Conjeturó independientemente la conjetura de Goldbach. En 1757 se graduó como senior wrangle. A finales de 1759 publicó el primer capítulo de \textit{Miscellanea Analytica}. En 1760 fue nombrado Profesor Lucaciano de Matemáticas. En 1762 publicó la \textit{Miscelánea analítica} completa, dedicada principalmente a la teoría de números y ecuaciones algebraicas. En 1763 fue elegido miembro de la Royal Society, siendo galardonado con la Medalla Copley en 1784, pero se retiró de la sociedad en 1795, después de haber cumplido los sesenta años. Obra:
}

\bioitem{bioGauss}{Carl Friedrich Gauss}{1777}{1855}{%(11.08.2024)
Matemático, físico y astrónomo alemán. Nació en Brunswick, Ducado de Brunswick-Wolfenbüttel, Sacro Imperio Romano Germánico. Fue un prodigio en matemáticas desde una edad temprana. Conocido como el "Príncipe de las Matemáticas", hizo contribuciones fundamentales a numerosas áreas, incluyendo la teoría de números, análisis, geometría diferencial, geofísica, electrostática, astronomía y óptica. Publicó su obra maestra, \textit{Disquisitiones Arithmeticae}, en 1801, que estableció su reputación como uno de los matemáticos más importantes de todos los tiempos. Obra:
}

\bioitem{bioEuler}{Leonhard Euler}{1707}{1783}{%(11.08.2024)
Matemático y físico suizo. Nació en Basilea, Suiza. Considerado uno de los matemáticos más prolíficos de todos los tiempos, sus trabajos abarcan una amplia gama de áreas, como la teoría de números, geometría, análisis matemático, mecánica y óptica. 
Fue el primero en introducir el concepto de función matemática y desarrolló muchas notaciones matemáticas modernas. 
Su obra incluye el desarrollo de la teoría de grafos, y su fórmula para poliedros, conocida como la fórmula de Euler, es una de las más célebres en la historia de la matemática. Obra:
}

\bioitem{bioNewton}{Isaac Newton}{}{?}{%(11.08.2024)
Físico, matemático y astrónomo inglés. Nació en Woolsthorpe, Lincolnshire, Inglaterra. Es considerado uno de los científicos más influyentes de todos los tiempos. Su obra más famosa, 
\textit{Philosophiæ Naturalis Principia Mathematica}, publicada en 1687, establece las leyes del movimiento y la ley de la gravitación universal, formando la base de la física clásica. 
Además, Newton hizo contribuciones significativas a las matemáticas, incluyendo el desarrollo del cálculo diferencial e integral, así como estudios en óptica. Obra:
}